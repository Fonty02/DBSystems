\section{Implementation}
\subsection{Types Definition}

\begin{lstlisting}
    CREATE OR REPLACE TYPE TeamTY AS OBJECT (
        Code VARCHAR2(20),
        Name VARCHAR2(50),
        Uptime INTEGER
    )
\end{lstlisting}

\begin{lstlisting}
    CREATE OR REPLACE TYPE FacilityTY AS OBJECT (
        Name VARCHAR2(50),
        Location VARCHAR2(100),
        Type VARCHAR2(20),
        EfficiencyScore NUMBER,
        MaxOutputEnergy NUMBER,
        SumOutputEnergy NUMBER,
        Team REF TeamTY)
\end{lstlisting}

\begin{lstlisting}
    CREATE OR REPLACE TYPE EmployeeTY AS OBJECT (
        FC VARCHAR2(20),
        Name VARCHAR2(50),
        Surname VARCHAR2(50),
        DoB DATE,
        DoH DATE,
        Manager CHAR(1),
        Team REF TeamTY
    )
\end{lstlisting}


\begin{lstlisting}
    CREATE OR REPLACE TYPE CustomerTY AS OBJECT (
        Code VARCHAR2(20),
        Name VARCHAR2(50),
        Surname VARCHAR2(50),
        Email VARCHAR2(100),
        Type VARCHAR2(20),
        DoB DATE
    )
\end{lstlisting}


\begin{lstlisting}
CREATE OR REPLACE TYPE AccountTY AS OBJECT (
    Code VARCHAR2(20),
    Customer REF CustomerTY
    )
\end{lstlisting}



\begin{lstlisting}
CREATE OR REPLACE TYPE ContractTY AS OBJECT (
        ID VARCHAR2(20),
        Active CHAR(1),
        Type VARCHAR2(20),
        Start_Date DATE,
        Cost NUMBER,
        EnergyPlan NUMBER,
        Duration NUMBER,
        Account REF AccountTY,
        Facility REF FacilityTY
    )
\end{lstlisting}


\begin{lstlisting}
    CREATE OR REPLACE TYPE FeedbackTY AS OBJECT (
        Code VARCHAR2(20),
        Message VARCHAR2(200),
        Score INTEGER,
        Account REF AccountTY ,
        Team REF TeamTY
    )
\end{lstlisting}

\subsection{Tables Definition}

\begin{lstlisting}
    CREATE TABLE Team OF TeamTY (
        Code PRIMARY KEY,
        CONSTRAINT chk_uptime CHECK (Uptime BETWEEN 6 AND 8),
        Name NOT NULL,
        Uptime NOT NULL
    )
\end{lstlisting}

We can see that simple constraints are handled with \textbf{check}

\begin{lstlisting}
    CREATE TABLE Facility OF FacilityTY (
        Name PRIMARY KEY,
        CONSTRAINT chk_facility_type CHECK (Type IN (''wind'', ''solar'', ''hydro'', ''nuclear'')),
        CONSTRAINT chk_max_output_energy CHECK (MaxOutputEnergy >= 0),
        CONSTRAINT chk_sum_output_energy CHECK (SumOutputEnergy >= 0),
        Location NOT NULL,
        Type NOT NULL,
        MaxOutputEnergy NOT NULL,
        EfficiencyScore NOT NULL,
        Team REFERENCES Team ON DELETE SET NULL
    )
\end{lstlisting}


\begin{lstlisting}
    CREATE TABLE Employee OF EmployeeTY (
        FC PRIMARY KEY,
        CONSTRAINT chk_manager CHECK (Manager IN (''Y'', ''N'')),
        Name NOT NULL,
        Surname NOT NULL,
        DoB NOT NULL,
        DoH NOT NULL,
        Manager NOT NULL,
        CONSTRAINT chk_doh_after_dob CHECK (DoH > DoB),
        Team REFERENCES Team ON DELETE SET NULL
    )
\end{lstlisting}


\begin{lstlisting}
    CREATE TABLE Customer OF CustomerTY (
        Code PRIMARY KEY,
        CONSTRAINT chk_customer_type CHECK (Type IN (''Commercial'', ''Residential'')),
        Name NOT NULL,
        Surname NOT NULL,
        Email NOT NULL,
        Type NOT NULL,
        DoB NOT NULL
    )
\end{lstlisting}



\begin{lstlisting}
    CREATE TABLE Account OF AccountTY (
        Code PRIMARY KEY,
        Customer NOT NULL REFERENCES Customer ON DELETE CASCADE
    )
\end{lstlisting}


\begin{lstlisting}
    CREATE TABLE Contract OF ContractTY (
        ID PRIMARY KEY,
        CONSTRAINT chk_contract_type CHECK (Type IN (''Commercial'', ''Residential'')),
        CONSTRAINT chk_energy_plan CHECK (EnergyPlan IN (2500, 4500, 10000)),
        CONSTRAINT chk_active CHECK (Active IN (''Y'', ''N'')),
        CONSTRAINT chk_cost CHECK (Cost >= 0),
        CONSTRAINT chk_duration CHECK (Duration >= 1),
        Start_Date NOT NULL,
        Cost NOT NULL,
        EnergyPlan NOT NULL,
        Duration NOT NULL,
        Active NOT NULL,
        Account NOT NULL REFERENCES Account ON DELETE CASCADE,
        Facility NOT NULL REFERENCES Facility ON DELETE CASCADE,
        Type NOT NULL
    )
\end{lstlisting}


\begin{lstlisting}
    CREATE TABLE Feedback OF FeedbackTY (
        Code PRIMARY KEY,
        CONSTRAINT chk_feedback_score CHECK (Score BETWEEN 1 AND 5),
        Message NOT NULL,
        Score NOT NULL,
        Account REFERENCES Account ON DELETE SET NULL,
        Team NOT NULL  REFERENCES Team ON DELETE CASCADE
    )
\end{lstlisting}

\subsection{Population procedure}

\begin{lstlisting}
    CREATE OR REPLACE PROCEDURE PopulateDatabase(
        p_num_customers IN NUMBER,
        p_num_accounts IN NUMBER,
        p_num_contracts IN NUMBER,
        p_num_feedbacks IN NUMBER
    ) IS
    BEGIN
    
        FOR i IN 1..50 LOOP
            INSERT INTO Team VALUES (
                TeamTY(
                    'Team' || TO_CHAR(i),
                    'TeamName' || TO_CHAR(i),
                    ROUND(DBMS_RANDOM.VALUE(6, 8))
                )
            );
        END LOOP;
    
    
    
        FOR team_id IN 1..50 LOOP
            FOR i IN 1..5 LOOP
                DECLARE
                    v_employee_code VARCHAR2(20);
                    v_team_code     VARCHAR2(20) := 'Team' || TO_CHAR(team_id);
                    v_dob           DATE;
                    v_doh           DATE;
                    v_manager       CHAR(1);
                BEGIN
    
                    IF i = 1 THEN
                        v_manager := 'Y';
                    ELSE
                        v_manager := 'N';
                    END IF;
    
                    v_employee_code := 'FC' || TO_CHAR((team_id - 1) * 5 + i);
                    
    
                    v_dob := ADD_MONTHS(SYSDATE, -ROUND(DBMS_RANDOM.VALUE(18*12, 60*12)));
                    
    
                    v_doh := v_dob + ROUND(DBMS_RANDOM.VALUE(18*365, 60*365));
    
                    INSERT INTO Employee VALUES (
                        EmployeeTY(
                            v_employee_code,
                            'Name' || v_employee_code,
                            'Surname' || v_employee_code,
                            v_dob,
                            v_doh,
                            v_manager,
                            (SELECT REF(t) FROM Team t WHERE t.Code = v_team_code)
                        )
                    );
                END;
            END LOOP;  
        END LOOP;
    
    
        FOR i IN 1..100 LOOP
            INSERT INTO Facility VALUES (
                FacilityTY(
                    'Facility' || TO_CHAR(i),
                    'Location' || TO_CHAR(i),
                    CASE MOD(i, 4)
                        WHEN 0 THEN 'wind'
                        WHEN 1 THEN 'solar'
                        WHEN 2 THEN 'hydro'
                        WHEN 3 THEN 'nuclear'
                    END,
                    100,
                    900000000,
                    0,
                    (SELECT REF(t) FROM Team t WHERE t.Code = 'Team' || TO_CHAR(CEIL(i/2)))
                )
            );
        END LOOP;
    
        
        
        
    
        FOR i IN 1..p_num_customers LOOP
            DECLARE
                v_dob DATE;
            BEGIN
               
                v_dob := ADD_MONTHS(SYSDATE, -ROUND(DBMS_RANDOM.VALUE(18*12, 60*12)));
    
                INSERT INTO Customer VALUES (
                    CustomerTY(
                        'Customer' || TO_CHAR(i),
                        'Name' || TO_CHAR(i),
                        'Surname' || TO_CHAR(i),
                        'email' || TO_CHAR(i) || '@example.com',
                        CASE WHEN MOD(i, 2) = 0 THEN 'Commercial' ELSE 'Residential' END,
                        v_dob
                    )
                );
            END;
        END LOOP;
    
        FOR i IN 1..p_num_accounts LOOP
            INSERT INTO Account VALUES (
                AccountTY(
                    'Account' || TO_CHAR(i),
                    (SELECT REF(c) FROM Customer c WHERE c.Code = 'Customer' ||  TO_CHAR(ROUND(DBMS_RANDOM.VALUE(1, p_num_customers))))
                )
            );
        END LOOP;
    
        FOR i IN 1..p_num_contracts LOOP
            DECLARE
                v_account_code VARCHAR2(20);
                v_customer_type VARCHAR2(20);
                v_start_date DATE;
                v_duration NUMBER;
                v_end_date DATE;
                v_active CHAR(1);
            BEGIN
                v_account_code := 'Account' || TO_CHAR(ROUND(DBMS_RANDOM.VALUE(1, p_num_accounts)));
    
                SELECT c.Type 
                INTO v_customer_type
                FROM Account a
                JOIN Customer c ON DEREF(a.Customer).Code = c.Code
                WHERE a.Code = v_account_code;
                v_start_date := SYSDATE;
                v_duration := ROUND(DBMS_RANDOM.VALUE(1, 12));
                v_end_date := ADD_MONTHS(v_start_date, v_duration);
                IF v_end_date >= SYSDATE THEN
                    v_active := 'Y';
                ELSE
                    v_active := 'N';
                END IF;
                INSERT INTO Contract VALUES (
                    ContractTY(
                        'Contract' || TO_CHAR(i),
                        v_active,  
                        v_customer_type, 
                        v_start_date,
                        DBMS_RANDOM.VALUE(100, 1000),
                        CASE MOD(i, 3)
                            WHEN 0 THEN 2500
                            WHEN 1 THEN 4500
                            WHEN 2 THEN 10000
                        END,
                        v_duration, 
                        (SELECT REF(a) FROM Account a WHERE a.Code = v_account_code),
                        (SELECT REF(f) FROM Facility f WHERE f.Name = 'Facility' || TO_CHAR(ROUND(DBMS_RANDOM.VALUE(1, 100))))
                    )
                );
            END;
        END LOOP;
    
        FOR i IN 1..p_num_feedbacks LOOP
            INSERT INTO Feedback VALUES (
                FeedbackTY(
                    'Feedback' || TO_CHAR(i),
                    'Message' || TO_CHAR(i),
                    ROUND(DBMS_RANDOM.VALUE(1, 5)),
                    (SELECT REF(a) FROM Account a WHERE a.Code = 'Account' || TO_CHAR(ROUND(DBMS_RANDOM.VALUE(1, p_num_accounts)))),
                    (SELECT REF(t) FROM Team t WHERE t.Code = 'Team' || TO_CHAR(ROUND(DBMS_RANDOM.VALUE(1, 50))))
                )
            );
        END LOOP;
    END;
\end{lstlisting}

\subsection{Trigger}

\begin{lstlisting}
    CREATE OR REPLACE TRIGGER trg_employee_manager
    FOR INSERT OR UPDATE ON Employee
    COMPOUND TRIGGER
        v_team REF TeamTY;
        v_count NUMBER;
        status CHAR(1);
    BEFORE EACH ROW IS
    BEGIN
        v_team := :new.Team;
        status:=:new.Manager;
    END BEFORE EACH ROW;
    AFTER STATEMENT IS
    BEGIN
        IF status = 'Y' THEN
            SELECT COUNT(*) INTO v_count
            FROM Employee
            WHERE (DEREF(Team)).Code = (DEREF(v_team)).Code
            AND Manager = 'Y';
    
            IF v_count > 1 THEN
                RAISE_APPLICATION_ERROR(-20001, 'Errore: Esiste già un manager per il team');
            ELSE
                DBMS_OUTPUT.PUT_LINE('Operazione eseguita con successo');
            END IF;
        END IF;
    END AFTER STATEMENT;
    END trg_employee_manager;
\end{lstlisting}

This trigger is used to check if there is already a manager in the team. If the new employee is a manager, the trigger checks if there is already a manager in the team. If there is, it raises an error, otherwise it allows the operation.

\begin{lstlisting}
    CREATE OR REPLACE TRIGGER trg_contract_type
    BEFORE INSERT OR UPDATE ON Contract
    FOR EACH ROW
DECLARE
     v_customer_type VARCHAR2(20);
BEGIN
     SELECT DEREF(a.Customer).Type
         INTO v_customer_type
         FROM Account a
         WHERE a.Code = (DEREF(:NEW.Account)).Code;

     IF :NEW.Type = v_customer_type THEN
            DBMS_OUTPUT.PUT_LINE('Operation Completed');
     ELSE
            RAISE_APPLICATION_ERROR(-20003, 'Error: Contract type and customer type mismatch');
     END IF;
END;
\end{lstlisting}

This trigger is used to check if the type of the contract is the same as the type of the customer. If they are different, it raises an error, otherwise it allows the operation.


\begin{lstlisting}
    CREATE OR REPLACE TRIGGER trg_deactivate_old_contracts
    FOR INSERT ON Contract
    FOLLOWS trg_contract_type
    COMPOUND TRIGGER
        v_account REF AccountTY;
        v_facility REF FacilityTY;
        v_contract_id VARCHAR2(20); 
        v_status CHAR(1);
        BEFORE EACH ROW IS
        BEGIN
            v_account := :NEW.Account;
            v_facility := :NEW.Facility;
            v_contract_id := :NEW.ID;
            IF :NEW.Active = 'N' THEN
                -- Check if contract's end date is in the past
                IF ADD_MONTHS(:NEW.Start_Date, :NEW.Duration) < SYSDATE THEN
                    DBMS_OUTPUT.PUT_LINE('You are inserting an expired contract');
                ELSE
                    DBMS_OUTPUT.PUT_LINE('Error: Contract is not active');
                    :NEW.Active := 'Y';
                END IF;
            END IF;
            v_status := :NEW.Active;
        END BEFORE EACH ROW;
        AFTER STATEMENT IS
        BEGIN
            IF v_status='Y' THEN
                FOR rec IN (
                    SELECT ID
                    FROM Contract
                    WHERE Active = 'Y'
                    AND Account = v_account
                    AND Facility = v_facility
                    AND ID <> v_contract_id
                ) LOOP
                    UPDATE Contract
                    SET Active = 'N'
                    WHERE ID = rec.ID;
    
                    DBMS_OUTPUT.PUT_LINE('Contract deactivated');
                END LOOP;
    
                DBMS_OUTPUT.PUT_LINE('Trigger trg_deactivate_old_contracts executed');
            END IF;
        END AFTER STATEMENT;
    
    END trg_deactivate_old_contracts;
    /
\end{lstlisting}

This trigger is used to deactivate old contracts when a new contract is inserted. It checks if the new contract is active and if the end date is in the past. If the end date is in the past, the new contract is inserted as inactive. Then, it deactivates all the other contracts of the same account and facility.


\begin{lstlisting}
    CREATE OR REPLACE TRIGGER trg_energy_plan
BEFORE INSERT OR DELETE OR UPDATE OF Active ON Contract
FOR EACH ROW
DECLARE
    v_facility_name Facility.Name%TYPE;
    v_current_sum NUMBER;
    v_max_output NUMBER;
BEGIN
    IF INSERTING OR UPDATING THEN
        SELECT f.Name INTO v_facility_name FROM Facility f WHERE REF(f) = :NEW.Facility;
    ELSE
        SELECT f.Name INTO v_facility_name FROM Facility f WHERE REF(f) = :OLD.Facility;
    END IF;
    SELECT SumOutputEnergy, MaxOutputEnergy 
    INTO v_current_sum, v_max_output 
    FROM Facility 
    WHERE Name = v_facility_name 
    FOR UPDATE;
    IF INSERTING THEN
        IF v_current_sum + :NEW.EnergyPlan > v_max_output THEN
            RAISE_APPLICATION_ERROR(-20001, 'SumOutputEnergy supera MaxOutputEnergy per la Facility ' || v_facility_name);
        ELSE
            UPDATE Facility 
            SET SumOutputEnergy = v_current_sum + :NEW.EnergyPlan 
            WHERE Name = v_facility_name;
        END IF;
    ELSIF DELETING THEN
        UPDATE Facility 
        SET SumOutputEnergy = v_current_sum - :OLD.EnergyPlan 
        WHERE Name = v_facility_name;
    ELSIF UPDATING THEN
        IF :OLD.Active = 'Y' AND :NEW.Active = 'N' THEN
            UPDATE Facility 
            SET SumOutputEnergy = v_current_sum - :OLD.EnergyPlan 
            WHERE Name = v_facility_name;
        END IF;
    END IF;
END trg_energy_plan;
\end{lstlisting}

This trigger is used to update the sum of the output energy of the facility when a contract is inserted, deleted or updated. It checks if the sum of the output energy exceeds the maximum output energy of the facility. If it does, it raises an error, otherwise it allows the operation.

\begin{lstlisting}
    CREATE OR REPLACE TRIGGER trg_efficiency_score
BEFORE UPDATE ON Facility
FOR EACH ROW
BEGIN
    IF :NEW.SumOutputEnergy = 0 THEN
        :NEW.EfficiencyScore := 100;
    ELSE
        :NEW.EfficiencyScore := 100 * :NEW.SumOutputEnergy / :NEW.MaxOutputEnergy;
    END IF;
END trg_efficiency_score;
\end{lstlisting}

This trigger is used to update the efficiency score of the facility when the sum of the output energy is updated. It calculates the efficiency score as the ratio between the sum of the output energy and the maximum output energy.

\begin{lstlisting}
    CREATE OR REPLACE TRIGGER trg_customer_age
AFTER INSERT OR UPDATE ON Customer
FOR EACH ROW
BEGIN
    IF MONTHS_BETWEEN(SYSDATE, :NEW.DoB) < 18*12 THEN
        RAISE_APPLICATION_ERROR(-20001, 'Error: Customer is a minor');
    END IF;
END;
\end{lstlisting}

\begin{lstlisting}
CREATE OR REPLACE TRIGGER trg_employee_age
AFTER INSERT OR UPDATE ON Employee
FOR EACH ROW
BEGIN
    IF MONTHS_BETWEEN(SYSDATE, :NEW.DoB) < 18*12 THEN
        RAISE_APPLICATION_ERROR(-20002, 'Error: Employee is a minor');
    END IF;
END;
\end{lstlisting}

These two triggers are used to check if the age of the customer or the employee is less than 18 years old. If it is, it raises an error, otherwise it allows the operation.


\subsection{Operations}

\begin{lstlisting}
    --------------------------------------------------------------------
    -- Procedure 1: Add a new customer to the database
    --------------------------------------------------------------------
    
    CREATE OR REPLACE PROCEDURE proc_register_customer (
        p_code      IN VARCHAR2,
        p_name      IN VARCHAR2,
        p_surname   IN VARCHAR2,
        p_email     IN VARCHAR2,
        p_type      IN VARCHAR2,  -- 'Commercial' o 'Residential'
        p_dob       IN DATE
    ) AS
    BEGIN
        INSERT INTO Customer
        VALUES (
            CustomerTY(p_code, p_name, p_surname, p_email, p_type, p_dob)
        );
        COMMIT;
    END;
    /
    
    --------------------------------------------------------------------
    -- Procedure 2: Add a new contract to the database
    --------------------------------------------------------------------
    CREATE OR REPLACE PROCEDURE proc_add_contract (
        p_contract_id    IN VARCHAR2,
        p_contract_type  IN VARCHAR2,  -- 'Commercial' o 'Residential'
        p_start_date     IN DATE,
        p_cost           IN NUMBER,
        p_energy_plan    IN NUMBER,    -- 2500, 4500 o 10000
        p_duration       IN NUMBER,    
        p_account_code   IN VARCHAR2,
        p_facility_name  IN VARCHAR2
    ) AS
    BEGIN
        INSERT INTO Contract
        VALUES (
            ContractTY(
                p_contract_id,
                'N',
                p_contract_type,
                p_start_date,
                p_cost,
                p_energy_plan,
                p_duration,
                (SELECT REF(a) FROM Account a WHERE a.Code = p_account_code),
                (SELECT REF(f) FROM Facility f WHERE f.Name = p_facility_name)
            )
        );
        COMMIT;
    EXCEPTION
        WHEN OTHERS THEN
           RAISE_APPLICATION_ERROR(-20011, 'Errore in proc_add_contract: ' || SQLERRM);
    END;
    /
    
    --------------------------------------------------------------------
    -- Procedure 3: Assign a facility to a team
    --------------------------------------------------------------------
    CREATE OR REPLACE PROCEDURE proc_assign_facility (
        p_facility_name IN VARCHAR2,
        p_team_code     IN VARCHAR2
    ) AS
        v_count_team NUMBER;
        v_count_facility NUMBER;
    BEGIN
        -- Check if the facility exists
        SELECT COUNT(*) INTO v_count_facility
        FROM Facility
        WHERE Name = p_facility_name;
    
        IF v_count_facility = 0 THEN
            RAISE_APPLICATION_ERROR(-20013, 'Error: Facility not found!');
        END IF;
    
        -- Check if the team exists
        SELECT COUNT(*) INTO v_count_team
        FROM Team
        WHERE Code = p_team_code;
    
        IF v_count_team = 0 THEN
            RAISE_APPLICATION_ERROR(-20014, 'Error: Team not found!');
        END IF;
        UPDATE Facility
        SET Team = (SELECT REF(t) FROM Team t WHERE t.Code = p_team_code)
        WHERE Name = p_facility_name;
    
        COMMIT;
    EXCEPTION
        WHEN OTHERS THEN
            RAISE_APPLICATION_ERROR(-20012, 'Errore in proc_assign_facility: ' || SQLERRM);
    END;
    /
    --------------------------------------------------------------------
    -- Function 4: Return the total energy output of a facility
    --------------------------------------------------------------------
    CREATE OR REPLACE FUNCTION func_get_facility_energy (
        p_facility_name IN VARCHAR2
    ) RETURN NUMBER AS
        v_total_energy NUMBER;
    BEGIN
         
        -- Return the total energy output of the facility
        SELECT f.SumOutputEnergy
          INTO v_total_energy
          FROM Facility f
         WHERE f.Name = p_facility_name;
        RETURN v_total_energy;
    EXCEPTION
        WHEN OTHERS THEN
           RAISE_APPLICATION_ERROR(-20013, 'Errore in func_get_facility_energy: ');
    END;
    /
    
    --------------------------------------------------------------------
    -- Procedure 5: Get the ranked facilities by efficiency score
    --------------------------------------------------------------------
    CREATE OR REPLACE PROCEDURE proc_get_ranked_facilities (
        p_cursor OUT SYS_REFCURSOR
    ) AS
    BEGIN
        OPEN p_cursor FOR
          SELECT Name, EfficiencyScore
            FROM Facility
           ORDER BY EfficiencyScore DESC;
    END;
    /
\end{lstlisting}